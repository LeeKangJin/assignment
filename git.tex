\documentclass[12pt]{article}
\usepackage{graphicx}
\usepackage[dvips]{epsfig}
\usepackage{amsfonts}
\usepackage{fancyhdr}
\usepackage{comment}

\usepackage[a4paper, top=2.5cm, bottom=2.5cm, left=2.2cm, right=2.2cm]%
{geometry}
\usepackage{times}
\usepackage{amsmath}
\usepackage{changepage}
\usepackage{amssymb}

\setcounter{MaxMatrixCols}{30}
\newtheorem{theorem}{Theorem}
\newtheorem{acknowledgement}[theorem]{Acknowledgement}
\newtheorem{algorithm}[theorem]{Algorithm}
\newtheorem{axiom}{Axiom}
\newtheorem{case}[theorem]{Case}
\newtheorem{claim}[theorem]{Claim}
\newtheorem{conclusion}[theorem]{Conclusion}
\newtheorem{condition}[theorem]{Condition}
\newtheorem{conjecture}[theorem]{Conjecture}
\newtheorem{corollary}[theorem]{Corollary}
\newtheorem{criterion}[theorem]{Criterion}
\newtheorem{definition}[theorem]{Definition}
\newtheorem{example}[theorem]{Example}
\newtheorem{exercise}[theorem]{Exercise}
\newtheorem{lemma}[theorem]{Lemma}
\newtheorem{notation}[theorem]{Notation}
\newtheorem{problem}[theorem]{Problem}
\newtheorem{proposition}[theorem]{Proposition}
\newtheorem{remark}[theorem]{Remark}
\newtheorem{solution}[theorem]{Solution}
\newtheorem{summary}[theorem]{Summary}
\newenvironment{proof}[1][Proof]{\textbf{#1.} }{\ \rule{0.5em}{0.5em}}

\newcommand{\Q}{\mathbb{Q}}
\newcommand{\R}{\mathbb{R}}
\newcommand{\C}{\mathbb{C}}
\newcommand{\Z}{\mathbb{Z}}

\begin{document}

\title{Assignment 01}
\author{Kang Jin Lee}
\date{\today}
\maketitle
\section{Git Tutorial}

	I try to make this document for understanding git command
  
\subsection{Git Configuration}

	If you use git first time insert your name and e mail for configuration
\begin{align}
git \quad config --global \quad user.name \quad"NAME" \\
git \quad config -- global \quad user.email\quad  EMAIL\quad   
\end{align}
 Enter your name to NAME in command (1) \\
 Enter your email in EMAIL command(2)

\begin{align}
\end{align}



\subsection{Git Start}
       you can make git repository as this command(4).
\begin{align}
git \quad init
\end{align}


\subsection{Git Modification}
		If you make a git repository you need next step.\\
		you add a first file as this command(5)
\begin{align}
git \quad add FILENAME
\end{align}
		and you can check your git state as this command(6)
\begin{align}
gir \quad status
\end{align}
		and if you chage the code of git repository. \\
		first you need to write down the comment 
\begin{align}
git \quad commit \\
git \quad commit -m \quad "comment "
\end{align}
if you have any git editor, command (7) is better. \\
but if you have not git editor just use the command (8)

\subsection{Git Management}
	you can download git repository to your local computer as this command(9)
\begin{align}
git \quad clone "HOSTADDRESS"
\end{align}
		If you commit the code than you can push your code to git repository
		\\ It will push in master branch		
		as this command (10)
\begin{align}
git \quad push \quad orgin \quad master
\end{align}
		and you can get the modication code to your local computer using this command (11)
\begin{align}
git \quad pull 
\end{align} 
and you can make the branch that it want to try in a diffrent direction than the orginal code
\begin{align}
git \quad checkout \quad -b \quad BRANCHNAME
\end{align}
		

\begin{figure}[!h]
%\centering
%\includegraphics[scale=0.2]{./a.eps}
\end{figure}
                       
  
  
  




\end{document}